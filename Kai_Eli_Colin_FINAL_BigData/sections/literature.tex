Due to the large amount of data that has been generated by  smart phones, there have been many papers published on human activity recognition. One group of researchers compared the performance of three different classifiers (Random Forest, Support Vector Machine, Na\"ive Bayes) and three different deep neural network architectures, namely Multilayer Perceptron, Deep Convolutional Neural Network, and Long-Short Term Memory on labeled accelerometer, gyroscope, magnetometer and electrocardiogram data \cite{masum2019human}. The researchers found that the deep neural networks performed much better than other machine-learning methods. Another group compared the performance of traditional machine learning methods (k-NN, SVM) to that of a residual neural network (ResNet) \cite{ferrari2019hand}. These researchers used a variety of datasets, including Motion Sense, UCI--HAR, and MobiAct. They used only the accelerometer data, and compared the classifier performance on the raw data vs hand-crafted features. For the Motion Sense dataset, the hand-crafted features improved the performance of the machine learned classifiers, but the ResNet applied to the raw data still achieved the highest accuracy. ResNets allow for skipping between network layers; these skips contain non-linearities, which the researchers claim allow them to better represent the non-linear relationships present in accelerometer data \cite{he2016deep}.

One theme we noticed in the literature was that researchers appeared to be in an accuracy arms race. Rather than focus on creating an algorithm that can function in a variety of real-world settings, they, by and large, aimed for the most accurate classifier within a single academic dataset. Perhaps no paper embodied this more than \cite{vavoulas2016mobiact}, which claimed an accuracy of 99.88\% on the MobiAct dataset using an intense period of trial and error with different window sizes, overlaps, and combinations of hand-crafted features for their neural network. 

This brings us back to our main goal: we do not wish to join the accuracy arms race. Instead we are aiming to develop a robust classifier that is not overfit to one specific academic dataset. Rather, we wish to successfully predict human activity across multiple academic datasets in order to create a method capable of functioning in real-world settings. 
