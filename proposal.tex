\documentclass[letterpaper]{article}
\usepackage[margin=1.5in]{geometry}
\usepackage{graphicx}
\usepackage{amsthm}
\usepackage{enumitem}
\usepackage{fourier}
\usepackage{physics}
\usepackage[USenglish]{babel}
\usepackage{microtype}
% info for header block in upper right hand corner
\author{Colin Adams\\
      Kai Kaneshina\\
      Eli Weissler  
        }
\title{\huge\textbf{Project Proposal}\\
\Large\textsc{Math 189r, Summer 2020}}
\date{\today}

\renewcommand{\labelenumi}{{(\alph{enumi})}}


\begin{document}
    \maketitle
    We are going to continue working on a project that we started in Prof.\ Gu's nonlinear data analytics class earlier this summer where we try and classify what physical activity a person is performing (e.g.\ walking, sitting, jogging) using the data recorded by their smartphone. We are going to expand our previous work by including multiple classifiers to distuingish more vigorous activities (e.g.\ jogging) versus more sedentary activities (such as sitting). Further, we are going to implement our own features (e.g.\ torsion and curvature as well as the frequency components) of the accelerometer data.  The interesting bit of this is to see if we can accurately classify activities across two large datasets. 

    \paragraph{Goal} Quite simply, our goal is to classify time-series accelerometer data from a cell phone and predict what activity the user is performing regardless of age, sex, etc. 

    \paragraph{Data} We are using data generated by the MotionSense Dataset which has attitude, gravity, userAcceleration, and rotationRate collected by an iPhone 6S. A total of 24 participants in a range of sex, age, weight, and height performed 6 activities in 15 trials in the same environment and conditions: downstairs, upstairs, walking, jogging, sitting, and standing. Unlike HMOG, we are not looking to identify personal trait attributes but rather are trying to group more collective behavior. Each of the participants kept the phone in their pocket as if they would during a normal day rather than other more controlled datasets we found (e.g.\ attaching the smartphone to the waist). The total number of data point is greater than 1.4 million. In addition to the MotionSense dataset, we will also use the MobiAct dataset which uses more than sixty participants doing the same activities.

    \paragraph{Stretch Goal} If time permits, we will take our own gyroscopic data on ourselves (using an iPhone 8) and determine its accuracy doing of classifying the activities that we ourselves are doing the same tasks of downstairs, upstairs, walking, jogging, sitting, and standing.

    \paragraph{Societal Context} This will be a good task to explore ways in which surveillance could be used on a population. For example, if an activity classifier is always running around in the background of a phone (similar to how HMOG authentication would work) then a government could not only know your location data but also know what you're doing while you're there.


\end{document}

